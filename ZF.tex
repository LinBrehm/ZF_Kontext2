\chapter{Textarten}
\section{Technischer Bericht}
Ein Technischer Bericht ist eine sachliche Beschreibung eines Konzepts, Experiments, Sachverhalts, Vorgangs, Produkts.\\
\\
\textbf{Kriterien:} Objektivität, Genauigkeit/Detailtreue, Nüchternheit im Sprachlichen\\
\\
\textbf{Anwendungsbeispiele:} Projektbericht, Produktbericht, Konzeptbericht, Innovationsbericht, Instruktionsbericht, Messbericht
%--------------------------------------------------------------------------------%
\subsection{Inhaltliche Bestandteile}
\begin{minipage}{0.48\textwidth}
	\begin{figure}[H]
		\includegraphics[width=\textwidth]{pictures/Technischer_Bericht_Bestandteile.png}
	\end{figure}
\end{minipage}\\
\textbf{Einleitung und Grundlagen}\\
• Auftrag, Ausgangslage und Ziele des Berichts \\
• Schilderung von Grundlagen, auf denen der Bericht basiert\\
\\
\textbf{Methodik und Ergebnisse}\\
• Beschreibung der angewandten Methoden, Fokus auf theoretische und technische Überlegungen\\
• Eingehen auf die ermittelten Resultate\\
• klare Strukturierung, klare Chronologie\\
• Visualisierungen verdichten das Geschriebene\\
\\
\textbf{Diskussion und Schlussfolgerung}\\
• Interpretation der Resultate\\
• Empfehlungen oder weiterführende Aufträge\\
• Ausblick\\
\\
\textbf{Vor dem Schreiben}\\
• Zielgruppe analysieren (Fachleute vs. Laien).\\
• Erwartungen der Beteiligten klären.\\
• Relevanz und Kontext des Berichts bedenken.\\
• Systematische, logische Darstellung ist essenziell.\\
\\
\textbf{Erarbeitungsprozess}\\
\textbf{1.Aufbau:}\\ 
Nehmen Sie eine inhaltliche Gliederung vor: \\
Auftrag/Ausgangslage/Ziel? Theoretische Überlegungen und Vorgehen? Resultate? Erklärung der Befunde? Fazit?\\
\\
\textbf{2.Strukturierung:}\\ 
Strukturieren Sie mit nummerierten Kapiteln. Setzen Sie Abschnitte und aussagekräftige Titel.\\
\\
\textbf{3.Visualisierung:}\\
Tabellen, Bilder, Diagramme einplanen.
%--------------------------------------------------------------------------------%
\subsection{Sprache}
\textbf{Sprache}\\
• Fachvokabular je nach Zielgruppe\\
• Sachlichkeit und Objektivität: keine Emotionen, Meinungsäusserungen und Kommentare\\
• Vermeidung des «Wir»-Stils\\
• Nüchternheit: keine Umschreibungen, Vergleiche oder Metaphern; wenige Adjektive\\
• Achtung: Zeitform und Zitate\\
\\
\textbf{Zeitform}\\
• \textit{Präteritum} oder \textit{Perfekt}, wenn über Geschehenes berichtet wird.\\
• \textit{Präsens} oder \textit{Futur}, wenn über Sachverhalte, die über das Ereignis hinaus Gültigkeit haben, oder über Informationen zur Gegenwart oder Zukunft berichtet wird.\\
\\
\textbf{Zitate}\\
• Mündliche Aussagen, die in einen Bericht integriert werden, sollten nicht in direkter Rede wiedergegeben werden.\\

\vfill\null


%--------------------------------------------------------------------------------%
%--------------------------------------------------------------------------------%
\section{Informierendes Schreiben}
\textbf{Textsorten:} Fachartikel, Nachricht, Kommentar, Medienmitteilung, Jahresbericht, Reportage
%--------------------------------------------------------------------------------%
\subsection{Leseverhalten}
\textbf{Text-/Bildelemente wahrnehmen}\\
Wir scannen Texte – gleich ob gedruckte oder online – meist von links oben nach rechts unten.\\
1. Bilder (grössere vor kleineren und grafisch auffälligere zuerst)\\
2. Haupttitel/Schlagzeile\\
3. Lead/Teaser\\
4. Zwischentitel\\
5. Bildlegenden\\ 
6. Haupttext\\
\\
\textbf{Lesetechniken}\\
• Vollständiges Lesen: Gründliches Lesen von A bis Z. \textit{Ziel:} Gesamtverständnis erhalten.\\
• Orientierendes Lesen: Überblick verschaffen. \textit{Ziel:} Eignung des Textes für weitere Nutzung abklären.\\
• Selektives Lesen: Ausgewählten Teil eines Ganzen lesen. \textit{Ziel:} Informationen aus einem Text beschaffen.\\
• Kursorisches Lesen: Texte überfliegen oder Speed Reading. \textit{Ziel:} möglichst viel Inhalt in kurzer Zeit aufnehmen.\\
• Studierendes Lesen Gründliches, langsames, z.T. wiederholendes Lesen. \textit{Ziel:} Gelesenes vertiefen und verarbeiten.\\
• Inspiratives Lesen Anregendes, animiertes Lesen. \textit{Ziel:} Unterhaltung.\\
\\
\textbf{Studienergebnisse Leseverhalten}\\
• \textit{Leseverständnis} ist bei der Lektüre auf Papier besser als ab Bildschirm. Ein Grund ist, dass Orientierung und Überblick auf Papier besser sind.\\
• Am Bildschirm neigen wir zu \textit{schnellem Lesen}.\\
• Dadurch fehlt die Zeit, \textit{Komplexes zu verstehen}. Wir nehmen meist nur die Fakten auf, ohne Schlüsse daraus zu ziehen und weiterzudenken.\\
• Im Endeffekt kann das dazu führen, dass wir unsere Fähigkeit nicht mehr trainieren, \textit{abstrakt zu denken} und \textit{kritisch zu analysieren}.

\begin{minipage}{0.48\textwidth}
	\begin{figure}[H]
		\includegraphics[width=\textwidth]{pictures/informierende_Texte_Struktur.png}
	\end{figure}
\end{minipage}\\


%--------------------------------------------------------------------------------%
\subsection{Inhaltliche Bestandteile}

\begin{minipage}{0.48\textwidth}
	\begin{figure}[H]
		\includegraphics[width=\textwidth]{pictures/informierende_Texte_Bausteine_1.png}
	\end{figure}
\end{minipage}\\

\vfill\null
\pagebreak
\textbf{Titel}\\
• Ist Eyecatcher\\
• Macht auf den Text aufmerksam und weckt die Lust am Lesen\\
• Bringt Thema oder Hauptaussage auf den Punkt\\
\\
\textbf{Übertitel/Untertitel}\\
• Steckt den Rahmen ab; ordnet ins Thema ein\\
• Ergänzt oder präzisiert den Haupttitel\\
\\
\textbf{Lead}\\
• Kurzzusammenfassung des Artikels\\
• Pointiert, keine Details\\
• Nüchterne Sprache, keine langen Sätze\\
• Umfang: 2–4 kurze Sätze, je nach Länge des Artikels; alles in einem Abschnitt\\
• Methode: «Küchenzuruf»\\
\\
\textbf{Lauftext}\\
• Prinzip der absteigenden Wichtigkeit\\
• W-Fragen\\

\begin{minipage}{0.48\textwidth}
	\begin{figure}[H]
		\includegraphics[width=\textwidth]{pictures/informierende_Texte_Bausteine_2.png}
	\end{figure}
\end{minipage}\\

\textbf{Zwischentitel}\\
• Ist Strukturierungsmittel\\
• Ermöglicht bessere inhaltliche Orientierung (besonders für Lesende, die nur abschnittsweise lesen)\\
• Greift inhaltlich die wichtigste Aussage des folgenden Absatzes auf\\
\\
\textbf{Quote}\\
• Zitierte Textpassage oder auf den Punkt gebrachte Aussage des Textes\\
• Grafisch hervorgehoben (Schriftgrösse/-farbe)\\

\textbf{Bild:}\\
\begin{minipage}{0.48\textwidth}
    \begin{figure}[H]
		\includegraphics[width=\textwidth]{pictures/informierende_Texte_Bild.png}
	\end{figure}
\end{minipage}\\

\textbf{Bildlegende}\\
Scharnierfunktion: beschreibt Bild, bezeichnet Menschen und verbindet zugleich mit dem Text\\
\\
\textbf{Tatsachenbild}\\
Nennt abgebildete Personen, Orte, Geschehnisse; bettet das Gezeigte zugleich in den Gesamtzusammenhang ein.\\

\vfill\null

\begin{minipage}{0.48\textwidth}
    \begin{figure}[H]
		\includegraphics[width=\textwidth]{pictures/informierende_Texte_Bildlegende_1.png}
	\end{figure}
\end{minipage}\\
\textbf{Symbolbild}\\
Bei Symbolbildern steht die Verbindung des Symbols zum Text im Zentrum, nicht die genaue Beschreibung des Abgebildeten.
\begin{minipage}{0.48\textwidth}
    \begin{figure}[H]
		\includegraphics[width=\textwidth]{pictures/informierende_Texte_Bildlegende_2.png}
	\end{figure}
\end{minipage}\\

\textbf{Formen von Legenden:}\\
\begin{minipage}{0.48\textwidth}
    \begin{figure}[H]
    \includegraphics[width=\textwidth]{pictures/informierende_Texte_Bildlegende.png}
    \end{figure}
\end{minipage}\\

\textbf{Infokasten:}\\
• Gefäss für weiterführende Hinweise, Kurzporträts, Tipps, Quellenangaben u.a.\\

%--------------------------------------------------------------------------------%
\subsection{Küchenzuruf und W-Fragen}
\textbf{Küchenzuruf:}\\
Eine Methode für Lead und Teaser\\
• Jeder Text hat eine Hauptaussage.\\
• Die Hauptaussage ist mit einem zusammenfassenden Zuruf vergleichbar, den jemand aufgrund einer gesehenen/gehörten Neuigkeit an eine nicht beteiligte Person in der Küche richtet.\\
Beispiel:\textit{«Du, Musk und Trump sind sich in die Haare geraten.»} \\
• Im Lead oder Teaser muss dieser Küchenzuruf stehen.\\
• Der Küchenzuruf bildet zugleich das Hauptthema des Textes, um ihn ist das Textgerüst aufgebaut.\\
\\
\textbf{W-Fragen:}\\
\textbf{Wer?}\\
Wer sind die Protagonisten in meinem Text?\\
Wer äusserte sich, beauftragte, handelte?\\
\textbf{Wann?}\\
Wann fand das Ereignis/Projekt statt?\\
\textbf{Wo?}\\
Wo fand das Geschehen statt?\\
\textbf{Was?}\\
Was ereignete sich?\\
Was war zu tun, zu erforschen, abzuklären?\\
\textbf{Warum?}\\
Warum ereignete sich etwas? Ursachen?\\
\textbf{Wie?}\\
Wie geschah es?\\
Wie wurde die Untersuchung durchgeführt?\\
Wie verliefen die einzelnen Phasen?\\
\textbf{Welche?}\\
Welche Auswirkungen sind bekannt?\\
Welche Konsequenzen sind zu ziehen?\\
\textbf{Woher?}\\
Woher stammen die Informationen (Quelle der Informationen)?\\
\\

\vfill\null
\pagebreak

\textbf{Beispiel Küchenzuruf:}\\
sda. Die Kantonspolizei Luzern sucht den Lenker eines grauen Mercedes. Der Mann war am 26. Januar in Horw in einen Unfall mit einem Skateboardfahrer verwickelt. Der Skateboardfahrer wurde verletzt. Der Skateboardfahrer fuhr gegen 13 Uhr auf dem Trottoir der Technikumstrasse in Richtung Pilatusmarkt, schreibt die Kantonspolizei in der Medienmitteilung. Gleichzeitig kam aus der Steinenstrasse ein grauer Mercedes gefahren und kollidierte auf dem Trottoir mit dem Skateboardfahrer. Dieser stürzte und zog sich mittelschwere Verletzungen zu. Der Fahrzeuglenker schaute kurz aus dem Personenwagen und entfernte sich von der Unfallstelle. Hinweise zum gesuchten Fahrzeuglenker oder zum Personenwagen bitte an die Kantonspolizei Luzern, Telefon 041 248 81 17.\\
\\
\textbf{W-Fragen I}\\
\textbf{Wer?} a) Kantonspolizei Luzern; b) Lenker eines grauen Mercedes\\
\textbf{Wann?} 26.1.2025, gegen 13 Uhr\\
\textbf{Wo?} Horw, Technikumstrasse, Trottoir\\
\textbf{Was?} Ein Auto kollidierte mit einem Skateboardfahrer\\
\textbf{Warum?} a) … kam es zum Unfall? Weil gleichzeitig zum Skateboarder, der auf der Technikumstrasse fuhr, ein Autofahrer von der Steinenstrasse her auf die Technikumstrasse einbog.\\
\textbf{Warum?} b) … sucht die Polizei den Lenker? Weil er sich von der Unfallstelle entfernte.\\
\\
\textbf{W-Fragen II}\\
\textbf{Wie?} Ein Skateboarder fuhr auf dem Trottoir der Technikumstrasse Richtung Pilatusmarkt, gleichzeitig kam aus der Steinenstrasse ein Autofahrer in die Technikumstrasse.\\
\textbf{Welche (Folgen)?} Gestürzter Skateboarder zog sich mittelschwere Verletzungen zu.\\
\textbf{Woher (Quelle)?} Medienmitteilung der Kantonspolizei Luzern\\
\textbf{Küchenzuruf:}\\
\textit{«Da hat einer einen Skateboardfahrer umgefahren und ist dann geflüchtet. Jetzt suchen sie ihn!»}\\
%--------------------------------------------------------------------------------%
%--------------------------------------------------------------------------------%
%--------------------------------------------------------------------------------%
\section{Protokolle}
\textbf{Funktion und Merkmale des Protokolls}\\
• Protokolle dokumentieren mündliche Beiträge schriftlich.\\
• Dienen als Gedächtnisstütze, Arbeitsinstrument, Nachweis oder juristische Grundlage.\\
• Inhalte: Stand der Dinge, Entscheidungen, nächste Schritte, Zuständigkeiten, Fristen.\\
• Gute Protokolle sind: sachlich, objektiv, klar, logisch gegliedert, verständlich – auch für Aussenstehende.\\
• Indirekte Rede: Wiedergabe von Aussagen, Meinungen, Vorschlägen etc., ohne diese direkt zu zitieren – aber dennoch inhaltlich korrekt und klar zuzuordnen.\\
\begin{center}
\small
\begin{tabular}{ll}
\textbf{Protokollart} & \textbf{Merkmale} \\
\hline
Wörtliches Protokoll & Vollständige,\\ 
                     & unveränderte Wiedergabe \\
Kurz-/Verlaufsprotokoll &  Konzentrierte,\\
                        & sachlogische Zusammenfassung\\
Ergebnisprotokoll & Nur Ergebnisse und Aufträge,\\
                  & keine Prozessdarstellung \\
Multiple-Choice-Protokoll & Rasterbasiert, standardisiert \\
Mindmap-Protokoll & Visuell, überblicksartig \\
Aktennotiz & Persönliche Gedächtnisstütze \\
\end{tabular}
\end{center}
%--------------------------------------------------------------------------------%
\subsection{Inhaltliche Bestandteile}
\textbf{Protokollrahmen}\\
• \textbf{Protokollkopf:} Was, wann, wo, wer, Traktandenliste.\\
• \textbf{Protokollschluss:} Ort, Datum, Verfasser, ggf. Verteiler, Beilagen.\\
\\
\textbf{Ablauf des Protokollierens}\\
• \textbf{Vorbereitung:} Protokollart klären, Unterlagen sammeln, Sitzordnung kennen.\\
• \textbf{Aufnahme:} Beiträge, Unterlagen und Unklarheiten festhalten.\\
• \textbf{Abfassen:} Rasch ausformulieren, Wiederholungen bündeln, wörtliche Zitate sparsam einsetzen.\\

%--------------------------------------------------------------------------------%
\subsection{Sprache}
\textbf{Indirekte Rede}\\
\textbf{Konjunktiv I} = „jemand hat etwas gesagt“ (Distanz wahren)\\
• Wird meistens verwendet.\\
• Signalisiert Distanz zur Aussage.\\
• Unabhängig von der Zeitform der ursprünglichen Aussage.\\
Beispiel:\\
Frau Meier sagt: „Das Projekt startet nächste Woche.“$\rightarrow$\\
Frau Meier sagt, das Projekt starte nächste Woche.\\

\vfill\null

\textbf{Konjunktiv II} = „etwas ist nicht real“ (Wunsch, Möglichkeit)\\
Wenn sich Konjunktiv I und Indikativ nicht unterscheiden \\
(„er sagt“ = „er sage“), benutzt man den Konjunktiv II.\\
Beispiel:\\
Herr Müller sagt: „Die Lösung funktioniert.“$\rightarrow$\\
Konjunktiv I = „funktioniert“ $\rightarrow$ identisch mit Indikativ\\
$\rightarrow$ Man nutzt Konjunktiv II:\\
$\rightarrow$ Herr Müller sagt, die Lösung funktioniere.

\begin{center}
\begin{tabular}{l|l|l}
\textbf{Modus} & \textbf{Bedeutung} & \textbf{Beispiel}\\
Indikativ & Realität/Tatsache & „Ich gehe jetzt nach Hause.“ \\
Konjunktiv I &  Indirekte Rede & „Er sagt, er gehe nach Hause.“\\
Konjunktiv II & Wunsch/Irrealität & „Ich wünschte, ich ginge ...“\\
\end{tabular}\\	   
\end{center}

\begin{minipage}{0.48\textwidth}
    \begin{figure}[H]
		\includegraphics[width=\textwidth]{pictures/protokolieren_indirekte_Rede_gemischter_Konjunktiv.png}
	\end{figure}
\end{minipage}\\

\textbf{Zeitform:} Meist Präsens, bei Rückblick Präteritum.\\
\\
\textbf{Aktiv bevorzugt:} Lebendige Sprache, ausser bei formellen Passivkonstruktionen.\\
\\
\textbf{Stil-Tipps:} Kurze, klare Sätze, wenige Adjektive, verständlich strukturieren („so viel wie nötig, so wenig wie möglich“).\\
%--------------------------------------------------------------------------------%
%--------------------------------------------------------------------------------%
\section{E-Mail}
\textbf{Grundsätzliches zur E-Mail im Berufsalltag}\\
• E-Mails sind effizient, aber schnell missverständlich.\\
• Sie ersetzen keine persönlichen Gespräche, besonders bei Konflikten oder heiklen Themen.\\
\\
\textbf{Regeln für sachliche Kommunikation:}\\
• Keine Konflikte per E-Mail austragen.\\
• Keine persönlichen Angriffe oder Gerüchte verbreiten.\\
• Persönliche Themen persönlich besprechen.\\
• Nie impulsiv auf provokante Mails reagieren.\\
\\
\textbf{Reaktionsverhalten}\\
• Nicht jede Mail erfordert sofortige Antwort.\\
• Zeitmanagement-Tipp: Feste Zeiten zum E-Mail-Checken.\\
\textbf{Antwortmatrix (nach David Bauer):}\\
• Wichtig + dringend: \textit{sofort reagieren}.\\
• Wichtig, nicht dringend: \textit{gelegentlich}.\\
• Unwichtig, dringend:  \textit{evtl. ignorieren}.\\
• Unwichtig, nicht dringend: \textit{ignorieren}.\\
\\
\textbf{Mailketten kritisch hinterfragen}\\
• Lange „AW“- oder „WG“-Ketten vermeiden.\\
• Nur nötige Informationen weiterleiten.\\
• Vorsicht bei vertraulichen Inhalten und Mail-Adressen.\\

%--------------------------------------------------------------------------------%
\subsection{Formale und sprachliche Anforderungen}
\textbf{Empfängerzeilen:}\\
• \textbf{An:} Hauptadressaten (Antwort erwartet).\\
• \textbf{Cc:} zur Information (keine Antwort nötig).\\
• \textbf{Bcc:} mit Bedacht einsetzen (z.B. Massenmails, Datenschutz).\\
\\
\textbf{Betreffzeile:}\\
• Klar, konkret, relevant.\\
• Keine Spam-Begriffe, keine Sonderzeichen.\\

\vfill\null
\pagebreak

\begin{center}
\small
\begin{tabular}{ll}
\textbf{Element} & \textbf{Hinweise} \\
\hline
Anrede & „Sehr geehrte…“, „Guten Tag…“, „Liebe…“\\
    &je nach Verhältnis. \\
Fliesstext &  Kurze Sätze, strukturierter Aufbau\\
    &(Einleitung, Hauptteil, Schluss).\\
Gruss & „Freundliche Grüsse“ ist Standard.\\
    & Vermeide veraltete oder blumige Formen.\\
Signatur & Vollständige Kontaktdaten,\\
    &keine übertriebenen Werbesprüche.\\
\end{tabular}\\	   
\end{center}
\textbf{Fliesstext}\\
• \textbf{Einleitung:} erläutert Ausgangslage; formuliert Anknüpfungspunkt eines laufenden Dialogs\\
• \textbf{Hauptteil:} enthält sämtliche relevanten Informationen, evtl. auch Links\\
• \textbf{Abschluss:} gibt einen kurzen Ausblick oder formuliert eine Würdigung\\
\\
\textbf{Sprache}\\
• Einfach, klar, höflich, korrekt.\\
• Leser:innen direkt ansprechen.\\
• Immer nochmals gegenlesen, bevor du sendest.\\
%--------------------------------------------------------------------------------%
%--------------------------------------------------------------------------------%
\section{Management Summary}
\textbf{Definition}\\
• Zusammenfassung eines Textes nach vorgegebenen inhaltlichen Kriterien.\\
• Fokussiert auf handlungsrelevante Aspekte, da es für Entscheidungsträger geschrieben wird.\\
• Vergleichbar mit Abstract in wissenschaftlichen Texten oder klassischen Zusammenfassungen.\\
\\
\textbf{Kriterien}\\
• Lesenden rasch einen Überblick über einen Text bieten.\\
• Laien das Wichtigste einer Expertise vermitteln.\\
• Entscheidungsträgern unter Nennung aller Kernaussagen ermöglichen, rasch beurteilen zu können, was zu tun ist.\\
\\
\textbf{Anwendungsbereich}\\
• in Berichten aller Art \\
• in Projektanträgen\\
• in Businessplänen\\
• in Studien \\
• in Marktanalysen \\
• in komplexen Offerten\\
\\
\textbf{Zielgruppe}\\
• \textbf{Entscheidungsträger:} Management Summary fasst die entscheidungsrelevanten Aspekte eines Handlungsbereichs zusammen.\\
• \textbf{Stakeholder:} Management Summary fasst Ergebnis einer in Auftrag gegebenen Untersuchung (z.B. in der Marktforschung) zusammen.\\
%--------------------------------------------------------------------------------%
\subsection{Inhaltliche Bestandteile}
• Basissatz\\
• Schilderung der Ausgangslage\\
• Informationen zum methodischen Vorgehen\\
• Ergebnisse\\
• Ausblick mit Handlungsempfehlung\\
\\
\textbf{Basissatz}\\
\begin{minipage}{0.4\textwidth}
    \begin{figure}[H]
		\includegraphics[width=\textwidth]{pictures/management_summary_basissatz.png}
	\end{figure}
\end{minipage}\\
\textbf{Ausgangslage}\\
• Einführungssatz/Basissatz\\
• Problemstellung\\
• Stand der Dinge, Grundlagen\\
• Fragestellungen, Ziele\\
\\
\textbf{Vorgehen}\\
• Methodik: Aufzeigen, wie Autor/in zu den Befunden kommt, auf welche Quellen er/sie sich stützt\\
• In Projektberichten: Projektaufbau (Beteiligte, Kosten, Termine u.a.) und Projektschritte\\
\\
\vfill\null

\textbf{Ergebnisse}\\
• Befunde und Hauptaussagen\\
• Auswertung der Zielerreichung\\
• erzielter Nutzen\\
\\
\textbf{Ausblick}\\
• Massnahmen\\
• Alternativen\\
• Folgerungen\\
• Handlungsempfehlungen\\
%--------------------------------------------------------------------------------%
\subsection{Formale Vorgaben}
\textbf{Beachten}\\
• \textbf{Platzierung:} am Anfang des Ausgangstextes; bildet kein eigenständiges Kapitel (trägt folglich keine Kapitelnummer)\\
• \textbf{Umfang:} hängt vom Ausgangstext ab; max. 5 Seiten\\
• \textbf{Aufbau:} mehrere Abschnitte mit Zwischentiteln\\
• \textbf{Text:} nennt Vorlage, auf die sich Summary bezieht (Basissatz); enthält Schlüsselbegriffe; Handlungsempfehlungen stützen sich auf Ergebnisse der Untersuchung\\
• \textbf{Visualisierung:} Integration von Tabellen/Infografiken sinnvoll\\
\\
\textbf{Vermeiden}\\
• Keine zusätzlichen Informationen: im Rahmen dessen bleiben, was der Ausgangstext an Informationen enthält\\
• Keine Definitionen: verwendete Begriffe müssen für Zielgruppe verständlich sein\\
• Keine Zitate: sich an Ergebnissen orientieren, nicht an Bestandsaufnahmen\\
• Keine Selbsteinschätzungen: konsequent objektiv bleiben\\
\\
\textbf{Sprache}\\
• sachlich, objektiv, wertfrei (kein «ich», «wir»)\\
• kurze, einfache Sätze\\
• kein Fachchinesisch\\
• einfache, prägnante, klar strukturierte Sprache\\
%--------------------------------------------------------------------------------%
\subsection{Managenment Summary vs. Abstract}
\begin{minipage}{0.48\textwidth}
    \begin{figure}[H]
		\includegraphics[width=\textwidth]{pictures/management_summary_vs_abstact.png}
	\end{figure}
\end{minipage}\\

 \vfill\null
 \pagebreak
%--------------------------------------------------------------------------------%
%--------------------------------------------------------------------------------%
\section{Textredaktion}
\textbf{Verständlich schreiben:} klar, strukturiert, korrekt und leserfreundlich.\\
\textbf{Gute Textredaktion kombiniert:} Sprache, Struktur, Technik und Tools\\
%--------------------------------------------------------------------------------%
\subsection{Verständlich schreiben}
\textbf{Allgemeine Tipps}\\
• \textbf{Ziel:} Leser sollen Inhalte einfach, klar und schnell verstehen.\\
• \textbf{Beim Schreiben Alternativen prüfen:} Wortwahl, Satzstruktur, Satzstellung.\\
• \textbf{Synonyme} nutzen (z.B. mit Thesaurus oder Online-Wörterbüchern).\\
%--------------------------------------------------------------------------------%
\subsection{Kürzen \& Vereinfachen}
• \textbf{Lange Formulierungen durch kürzere ersetzen:}\\
\textit{„zur Anwendung bringen“ $\rightarrow$ „anwenden“}\\
• \textbf{Redundanzen vermeiden:}\\
\textit{„die durchgeführte Untersuchung“ $\rightarrow$ „die Untersuchung“}\\
%--------------------------------------------------------------------------------%
\subsection{Satzbau optimieren}
\textbf{Kurze, einfache Sätze}\\
• Durchschnittlich ca. 15 Wörter pro Satz.\\
• Pro Satz eine zentrale Information.\\
\\
\textbf{Satzstellung beachten}\\
• Normale Reihenfolge: Subjekt – Verb – Objekt. \\
\textit{„Die Gemeinde prüft den Zustand.“}\\
\\
\textbf{Verkettungen vermeiden}\\
• Lange Attribut-Ketten auflösen.$\rightarrow$\\
Besser: In mehrere kurze Sätze umformulieren.\\
\\
\textbf{Nebensätze vereinfachen}\\
• Keine verschachtelten Sätze mit eingebetteten Nebensätzen.\\
• Klare Struktur: Zuerst Ursache, dann Wirkung.\\
\\
\textbf{Aktiv statt Passiv}\\
• Aktiv: \textit{„Die Techniker prüfen die Anlage.“}\\
• Passiv: „Die Anlage wird geprüft.“$\rightarrow$\\
Aktiv ist direkter, klarer und lebendiger.
%--------------------------------------------------------------------------------%
\subsection{Rechtschreibung \& Zeichensetzung}
\textbf{Fehlertoleranz \& Kontrolle}\\
• Viele Fehler = negativ für den Eindruck.\\
• Texte immer gegenlesen lassen.\\
• Eigene Schwächen kennen und gezielt prüfen.\\
• Tipp: Text laut vorlesen für bessere Fehlererkennung.\\
\\
\textbf{Hilfsmittel nutzen}\\
• Tools wie Duden.de, Rechtschreibprüfung, KI, Wörterbücher.\\
• Nicht blind auf Google verlassen – vertrauenswürdige Quellen bevorzugen.\\
• Eigene Fehlerliste anlegen für wiederkehrende Probleme.\\
%--------------------------------------------------------------------------------%
\subsection{Schreibtools \& KI}
• KI-Tools können Standardtexte (z.B. Wetterberichte) gut generieren.\\
• Sie sind schnell, günstig, aber:\\
Nur so gut wie die Eingaben (Prompts).\\
Ethische Fragen: Quellenangabe, Urheberrecht, Halluzinationen.\\
• Für Studium und Beruf: KI als Werkzeug, nicht als Ersatz für eigenes Denken.\\
• Nützliche Tools: Rechtschreibprüfung, Zusammenfassung, Übersetzung, Kollaboration.
%--------------------------------------------------------------------------------%
%--------------------------------------------------------------------------------%
%--------------------------------------------------------------------------------%

\chapter{Kommunikationstheorie}
\textbf{Kommunikation}\\
• Zwei Menschen schauen sich gegenseitig an\\
• Zwei Menschen sprechen miteinander\\
• Sie schreiben jemandem eine Notiz\\
• Sie lesen Zeitung\\
• Sie schreiben jemandem ein E-Mail\\
• Sie bloggen, twittern, chatten und liken\\
\\
\textbf{keine Kommunikation}\\
Eine Gewitterwolke sendet dir klare Signale, dass es regnen wird. Die Gewitterwolke reagiert jedoch nicht auf deine Aktion, somit regnet sie auch, wenn du den Regenschirm nicht dabei hast.

\vfill\null

%--------------------------------------------------------------------------------%
%--------------------------------------------------------------------------------%
\section{5 Axiome der Kommunikation}
\begin{minipage}{0.4\textwidth}
    \begin{figure}[H]
		\includegraphics[width=\textwidth]{pictures/komunikations_theorie_5_axiome.png}
	\end{figure}
\end{minipage}\\


%--------------------------------------------------------------------------------%
%--------------------------------------------------------------------------------%
\section{Kommunikations-Ebenen}
\subsection{Verbal}
\textbf{Verbale Kommunikation}\\
• Bezieht sich auf die gesprochene oder geschriebene Kommunikation\\
• Spricht den Hörsinn (Gesprochenes) oder Sehsinn (Geschriebenes) des Publikums an\\
\\
\textbf{Gelungene verbale Kommunikation}\\
• auf Zielpublikum ausgerichtet\\
• freies, flüssiges und gewandtes Sprechen\\
• verständliche Sprache\\
\\
\textbf{Verbale Kommunikation im Referat}\\
• \textbf{Einstieg:} Publikum abholen, ins Thema einstimmen, Programm/Ziele vorstellen\\
• \textbf{Hauptteil:} roter Faden, Spannungsbogen erzeugen\\
• \textbf{Schluss:} Zusammenfassung, Ausblick, neue Fragen\\
• \textbf{Metakommunikation} (durch Präsentation führen):\\
\textit{«Im Folgenden zeige ich Ihnen eine Zusammenstellung, aus der ersichtlich wird …»}\\
\textit{«Aus dem Gesagten lassen sich drei Forderungen ableiten: Erstens …, zweitens …, drittens …»}
%--------------------------------------------------------------------------------%
\subsection{Nonverbal}
\textbf{Nonverbale Kommunikation}\\
• Bezieht sich auf sämtliche Kommunikation, die nicht verbal (gesprochen/geschrieben) erfolgt: Mimik, Gestik, Blickkontakt usw.\\
• Spricht Beobachtungssinn des Publikums an\\
\\
\textbf{Gelungene nonverbale Kommunikation (Referat)}\\
• dem Publikum zugewandt\\
• natürliches, sicheres, authentisches und gepflegtes Auftreten\\
• Mimik und dosierte Gestik\\
• Blickkontakt suchen und variieren Negativbeispiele von Blickkontakt-Typen:\\
\textit{«Heilige»; «Pilzsucher»; «Hypnotiseure»; «Manuskript-Kleber»; «Segment-Blicker»}\\
• Bewegung im Raum (Proxemik)
%--------------------------------------------------------------------------------%
\subsection{Paraverbal}
\textbf{Paraverbale Kommunikation}\\
• Bezieht sich im Mündlichen auf die Eigenschaften der Stimme und auf das Sprechverhalten des/der Präsentierenden\\
• Bezieht sich im Schriftlichen auf das Schriftbild\\
• Spricht den Hörsinn (Gesprochenes) oder Sehsinn (Schriftliches) des Publikums an\\
\\
\textbf{Gelungene paraverbale Kommunikation}\\
• deutliche Aussprache\\
• Sprechtempo; Sprachrhythmus\\
• Modulation (keine Monotonie)\\
• Pausen (Wirkpausen, Denkpausen)\\
• Emotionalität (Bezug zum Thema)\\
• Ausdruck der Handschrift oder Wahl der Schriftart\\
• Schriftgrössen, Auszeichnungen

\vfill

\subsection{Metakommunikation}
Als Metakommunikation bezeichnet man die Kommunikation, die sich mit Kommunikation befasst. Sich über Kommunikation unterhalten.\\
\\
\textbf{Beispiel}\\
Du sagst in einem Streit:\\
\textit{„Ich finde, wir reden gerade aneinander vorbei.“}\\
Dann sprichst du nicht über das Thema des Streits, sondern über die Art, wie ihr streitet – das ist Metakommunikation.
%--------------------------------------------------------------------------------%
%--------------------------------------------------------------------------------%
\section{Organon Model}
Karl Bühler (1934)\\
Sprache als Werkzeug (griech. Organon), mit dem wir Dinge mitteilen.\\
\\
\textbf{Darstellungsebene:} Thema und Sachinhalt werden übermittelt.\\ 
\\
\textbf{Ausdrucksebene:} Der Sprechende/Schreibende gibt (indirekt) seine Haltung preis.\\
\\
\textbf{Appellebene:} Beim Hörenden/Lesenden soll eine Reaktion ausgelöst werden.\\
\\
\textbf{Beispiel}\\
Ein Mitarbeiter sagt zu seinem Arbeitskollegen:\\
\textit{«Es hat kein Papier mehr im Drucker.»}\\
• Darstellungsebene: \textit{Das Papierfach des Druckers ist leer.}\\
• Ausdrucksebene: \textit{Ich brauche Papier zum Kopieren. / Jemand muss nachfüllen, ich tue es nicht.}\\
• Appellebene: \textit{Füll Papier nach.}\\
\begin{minipage}{0.48\textwidth}
    \begin{figure}[H]
		\includegraphics[width=\textwidth]{pictures/komunikations_theorie_organon_model.png}
	\end{figure}
\end{minipage}\\


%--------------------------------------------------------------------------------%
%--------------------------------------------------------------------------------%
\section{Vier-Ohren Model}
F. Schulz von Thun\\
Jede Nachricht/Zeichenfolge enthält vier Botschaften.\\
\\
\textbf{Sachebene:} Was an Sachverhalten vermittelt wird.\\
\\
\textbf{Selbstkundgabe:} Was der Sender über sich vermittelt.\\
\\
\textbf{Beziehungsebene:} Wie der Sender zum Empfänger steht.\\
\\
\textbf{Appellebene:} Zu was der Empfänger bewegt werden soll.\\
\\
\textbf{Beispiel}\\
Ein Mitarbeiter sagt zu seinem Arbeitskollegen:\\
\textit{«Es hat kein Papier mehr im Drucker.»}\\
• Sachebene: \textit{Das Papierfach des Druckers ist leer.}\\
• Selbstkundgabe: \textit{Ich brauche Papier zum Kopieren. / Jemand muss nachfüllen, ich tue es nicht.}\\
• Beziehungsebene: \textit{Mitarbeiter fühlt sich dem Arbeitskollegen übergeordnet.}\\
• Appellebene: \textit{Füll Papier nach.}\\

\begin{minipage}{0.48\textwidth}
    \begin{figure}[H]
		\includegraphics[width=\textwidth]{pictures/komunikations_theorie_4_ohren_model.png}
	\end{figure}
\end{minipage}\\

\vfill\null